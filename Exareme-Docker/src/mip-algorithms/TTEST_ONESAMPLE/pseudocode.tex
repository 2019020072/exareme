\documentclass[10pt]{article}
\usepackage[inner = 2.0cm, outer = 2.0cm, top = 2.0cm, bottom = 2.0cm]{geometry}
\usepackage{mathtools}
\usepackage{graphicx}
\usepackage{amsmath}
\usepackage{algorithm}
\usepackage{algpseudocode}

\DeclareMathOperator*{\mean}{mean}
\DeclareMathOperator*{\std}{std}

\begin{document}
\begin{algorithm}
        \caption{\underline{\textsc{One Sample T-test }}}
        \label{naivebayes_train}
        \begin{algorithmic}[1] % The number tells where the line numbering should start
                \Procedure{Local1}{${x_j^{(l)}}, hypothesis$} \Comment{run for $l = 1, \dots, L$}
					\State $n^{(l)} \gets \text{count}^{(l)}(x_j^{(l)})$
                    \State $s_x^{(l)} \gets \sum_j x_j^{(l)}$
                    \State $s_{xx}^{(l)} \gets \sum_j {(x_j^{(l)})^2}$
                    \State \textbf{return} $(n^{(l)}, s_x^{(l)}, s_{xx}^{(l)})$
                \EndProcedure
                \Procedure{Global}{\{$n^{(l)}, s_x^{(l)}, s_{xx}^{(l)}\}, \mu$}
                     \State $n \gets \sum_l n^{(l)}$
                     \State $s_x \gets \sum_l s_x^{(l)}$
                     \State $s_{xx} \gets \sum_l s_{xx}^{(l)}$
                     \State Compute and output statistics, p-values
                \EndProcedure
        \end{algorithmic}
\end{algorithm}
\end{document}
